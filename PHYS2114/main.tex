\documentclass[12pt,english]{article}
\usepackage[a4paper,bindingoffset=0.2in,%
            left=1in,right=1in,top=1in,bottom=1in,%
            footskip=.25in]{geometry}
\usepackage{amsfonts, amsmath, amssymb, physics}
\usepackage[utf8]{inputenc}
\usepackage[english]{babel}
\usepackage{indentfirst}
\usepackage[backend=biber, citestyle=nature]{biblatex}
\usepackage[nottoc]{tocbibind}
\usepackage{halloweenmath}
\usepackage{calligra}
\usepackage{esint}
\usepackage{graphicx}
\usepackage{hyperref}
\usepackage{float}
\graphicspath{{./images/}}

%%%%%%%%%%%%%%%%%%%%% PREAMBLE %%%%%%%%%%%%%%%%%%%%%%%%
\setlength{\parindent}{2em}
\setlength{\parskip}{0em}

\newcommand{\dmr}[1]{\, \mathrm{d}#1} %command for the straight d (hehe)
\newcommand{\intt}[2]{\int_{#1}^{#2}} %easy integral command


\newtheorem{definition}{Definition}[section]
\newtheorem{theorem}{Theorem}[section]
\newtheorem{remark}{Remark}[subsubsection]

\numberwithin{equation}{subsection}

\addbibresource{ref.bib}


\DeclareMathAlphabet{\mathcalligra}{T1}{calligra}{m}{n}
\DeclareFontShape{T1}{calligra}{m}{n}{<->s*[2.3]callig15}{}
\newcommand{\curly}[1]{\ensuremath{\mathcalligra{#1}}}

\let\oldhat\hat
\renewcommand{\vec}[1]{\mathbf{#1}}
\renewcommand{\hat}[1]{\oldhat{\mathbf{#1}}}
\newcommand{\ar}[1]{\overset{\rightharpoonup}{#1}}
%%%%%%%%%%%%%%%%%%%%%%%%%%%%%%%%%%%%%%%%%%%%%%%%%%%%%%%

\title{PHYS2114 Cheat Sheet}
\date{June 2021}
\author{Joey Liang}
\begin{document}
\maketitle
\newpage
\tableofcontents
\newpage
\section{Mathematics Preliminary}
\subsection{IMPORTANT NOTE}
The following notations are taken from Griffths' `Introduction to Electromechanics'.\cite{Griffiths:611579}

Where as the line, area and volume integral elements are the following;
\begin{itemize}
    \item Line integral element: $\dmr{\vec{l}}.$
    \item Area integral element: $\dmr{\vec{a}}.$ 
    \item Volume integral element: $\dmr{\tau}.$
\end{itemize}

\par\noindent\rule{\textwidth}{0.4pt}

\subsection{Cartesian Coordinates}
The line and volume integral elements are
\begin{align*}
    \dmr{\vec{l}} &= \hat{x}\dmr{x} + \hat{y}\dmr{y} + \hat{z}\dmr{z}, & \dmr{\tau} = \dmr{x}\dmr{y}\dmr{z}.
\end{align*}

And the following are some common operators
\begin{align*}
    &\text{Gradient:} &\grad t  &= \pdv{t}{x}\hat{x} + \pdv{t}{y}\hat{y} + \pdv{t}{z}\hat{z},\\[1em]
    %
    &\text{Divergence:} &\grad\cdot\vec{v} &= \pdv{v_x}{x} + \pdv{v_y}{y} + \pdv{v_z}{z},\\[1em]
    %
    &\text{Curl:} &\grad \cross \vec{v} &= \left(\pdv{v_z}{y}-\pdv{v_y}{z}\right)\hat{x} + \left(\pdv{v_x}{z}-\pdv{v_z}{x}\right)\hat{y} + \left(\pdv{v_y}{x}-\pdv{v_x}{y}\right)\hat{z},\\[1em]
    %
    &\text{Laplacian:} &\laplacian t &= \pdv[2]{t}{x} + \pdv[2]{t}{y} +\pdv[2]{t}{z}. 
\end{align*}

\par\noindent\rule{\textwidth}{0.4pt}
\subsection{Cylindrical Coordinates}
The conversion of Cartesian to Cylindrical coordinates\cite{noauthor_cylindrical_2021} are 
\begin{align*}
    x &= r \cos \phi, & y &= r \sin\phi, & z &= z,
\end{align*}
note that $\phi \in [0,2\pi].$

And the surface integral element with radius $r$ constant is
\[
    \dmr{\vec{S}}    = r \dmr{\phi}\dmr{z}
\]    

The line and the volume integral elements are
\begin{align*}
    \dmr{\vec{l}} &= \hat{s}\dmr{s} + s\hat{\phi}\dmr{\phi}+\hat{z}\dmr{z}, & \dmr{\tau} = s\dmr{s}\dmr{\phi}\dmr{z}.
\end{align*}

The following are some common vector operators
\begin{align*}
    &\text{Gradient:} &\grad t  &= \pdv{t}{s}\hat{s}+\frac{1}{s}\pdv{t}{\phi}\hat{\phi}+\pdv{t}{z}\hat{z},\\[1em]
    %
    &\text{Divergence:} &\grad\cdot\vec{v} &= \frac{1}{s}\pdv{}{s}(sv_s)+\frac{1}{s}\pdv{v_\phi}{\phi}+\pdv{v_z}{z},\\[1em]
    %
    &\text{Curl:} &\grad \cross \vec{v} &= \left(\pdv{v_z}{y}-\pdv{v_y}{z}\right)\hat{x} + \left(\pdv{v_x}{z}-\pdv{v_z}{x}\right)\hat{y} + \left(\pdv{v_y}{x}-\pdv{v_x}{y}\right)\hat{z},\\[1em]
    %
    &\text{Laplacian:} &\laplacian t &= \frac{1}{s}\pdv{}{s}\left(s\pdv{t}{s}\right)+\frac{1}{s^2}\pdv[2]{t}{\phi}+\pdv[2]{t}{z}. 
\end{align*}
\par\noindent\rule{\textwidth}{0.4pt}

\subsection{Spherical Coordinates}
Note that $\theta\in[0,\pi]$ denotes the angle between z-axis and the vector of interest, and that $\phi\in[0, 2\pi]$ denotes the angle between x-axis and the projection of the vector of interest on to the xy-plane.\cite{noauthor_cylindrical_2021}

The conversion of Cartesian to Spherical coordinates are shown as the follows
\begin{align*}
    x &= r \sin(\theta)\cos(\phi), & y &= r \sin(\theta)\sin(\phi), & z &= r \cos(\theta).\\
\end{align*}

And the line and volume integral elements are
\begin{align*}
    \dmr{\vec{l}} &= \hat{r}\dmr{r} + r\hat{\theta}\dmr{\theta} + r  \sin(\theta) \hat{\phi} \dmr{\phi}, & \dmr{\tau} = r^2 \sin(\theta) \dmr{r} \dmr{\theta} \dmr{\phi}.
\end{align*}
with the surface integral with the radius $r$ constant is
\[
    \dmr{\vec{S}} = \sin\theta \dmr{\theta} \dmr{\phi}.
\]

The following are the common operators
\begin{align*}
    &\text{Gradient:} &\grad t  &= \hat{r}\pdv{t}{r} + \frac{\hat{\theta}}{r}\pdv{t}{\theta} + \frac{\hat{\phi}}{r \sin(\theta)}\pdv{t}{\phi},\\[1em]
    %
    &\text{Divergence:} &\grad\cdot\vec{v} &= \frac{1}{r^{2}} \frac{\partial}{\partial r}\left(r^{2} v_r \right)+\frac{1}{r \sin \theta} \frac{\partial}{\partial \theta}\left(v_{\theta} \sin \theta\right)+\frac{1}{r \sin \theta} \frac{\partial v_{\phi}}{\partial \phi},\\[1em]
    %
    &\text{Curl:} &\grad \cross \vec{v} &= \frac{1}{r\sin\theta}\left(\pdv{}{\theta}\sin(\theta) v_\phi- \pdv{v_\theta}{\phi}\right)\hat{r}\\
    &&&+\frac{1}{r}\left(\frac{1}{\sin\theta}\pdv{v_r}{\phi}-\pdv{}{r}  r v_\phi  \right)\hat{\theta}+ \frac{1}{r}\left(\pdv{}{r}rv_\theta-\pdv{v_r}{\theta}\right)\hat{\phi},\\[1em]
    %    
    &\text{Laplacian:} &\laplacian t &= \frac{1}{r^2}\pdv{}{r}\left(r^2\pdv{t}{r}\right)+ \frac{1}{r^2\sin\theta}\pdv{}{\theta}\left(\sin\theta\pdv{t}{\theta}\right) +\frac{1}{r^2 \sin^2 \theta}\pdv[2]{t}{\phi}.
\end{align*}

\par\noindent\rule{\textwidth}{0.4pt}

\subsection{Gauss' Divergence Theorem}
Suppose $V$ is a subset of $\mathbb{R}^n$ (in the case of $n=3$, $V$ represents a volume in three-dimensional space) which is compact and has a piecewise smooth boundary $S$ (also indicated with $\partial V = S$). If $\vec{F}$ is a continuously differentiable vector filed defined on a neighbourhood of $V$, then:
\[
    \iiint_{V} \left(\grad \cdot \vec{F}\right)\dmr{V} = \oiint_{S} \left(\vec{F}\cdot\hat{n}\right)\dmr{S}.
\]
\par The left side is a volume integral over the volume $V$, the right side is the surface integral over the boundary of the volume $V$. The closed manifold $\partial V$ is oriented by outward-pointing normal, and $\vec{n}$ is the outward pointing normal at each point on the boundary $\partial V$. ($\dmr{\vec{S}}$ may be used as a shorthand for $\vec{n}\dmr{S}$.) In terms of the intuitive description above, the left-hand side of the equation represents the total of the sources in the volume $V$, and the right-hand side represents the total flow across the boundary $S$.\cite{noauthor_divergence_2021}

\par\noindent\rule{\textwidth}{0.4pt}

\subsection{Stokes' Theorem}
Suppose we have a boundary $\partial \Sigma = S$ that bounds the surface $\Sigma$ with $\vec{F}$ defined in $\Sigma$, then
\[
    \iint_{\Sigma} (\grad \cross \vec{F})\cdot\vec{n}\dmr{S} = \oint_{S} \vec{F}\cdot \dmr{\vec{l}}.
\] 
\begin{figure}[h]
    \centering
    \includegraphics[width =0.5\textwidth]{1200px-Stokes'_Theorem.svg.png}
    \caption{A visual representation for the Stokes' theorem\cite{noauthor_stokes_2021}}
\end{figure}

\par\noindent\rule{\textwidth}{0.4pt}
\par\noindent\rule{\textwidth}{0.4pt}
\section{Electrostatics}
\subsection{The Holy Trinity}
\begin{align}
    &V = \frac{1}{4\pi\varepsilon_0}\iiint_{\mathcal{V}} \frac{\rho}{\curly{r}}\dmr{\tau}\\[1em]
    %%
    &\laplacian V = -\frac{\rho}{\varepsilon_0}\\[1em]
    %%
    &E = - \grad V\\[1em]
    %%
    &V = -\int \vec{E}\cdot \dmr{\vec{l}}\\[1em]
    %%
    &\vec{E} = \frac{1}{4\pi\varepsilon_0}\iiint_{\mathcal{V}} \frac{\hat{\curly{r}}}{\curly{r}^2}\rho\dmr{\tau}\\[1em]
    %%
    &\grad \cdot E = \frac{\rho}{\varepsilon_0}; \;\;\; \grad \cross E = 0
\end{align}
\begin{figure}[h]
    \centering
    \includegraphics[width =0.5\textwidth]{holytrinity.png}
    \caption{The Griffith's holy trinity of electrostatics\cite{Griffiths:611579}}
\end{figure}

\par\noindent\rule{\textwidth}{0.4pt}
\subsection{Electrostatic Boundary Conditions}
\begin{align}
    \hat{n}_2 \cdot [\vec{E}_1-\vec{E}_2] &= \frac{\sigma}{\varepsilon_0}\\[0.5em]
    \hat{n}_2 \cross [\vec{E}_1-\vec{E}_2] &= 0\\[0.5em]
    V_1 - V_2 &= 0
\end{align}
Note that here the ``1" and ``2" just refer to the different sides of the interface.

\par\noindent\rule{\textwidth}{0.4pt}
\subsection{Work and Energy in Electrostatics}
Three important equations for Energy in work and energy in electro statics:
\par For discrete charges that are not very closed together,
\begin{equation}
    W = \frac{1}{2} \sum_{i = 1}^{n}q_i V(\vec{r}_i).
\end{equation}
\par For volume charge density $\rho$, 
\begin{equation}
    W = \frac12 \iiint_{\mathcal{V}} \rho V \dmr{\tau}.
\end{equation}
\par And even more simply, we can have,
\begin{equation}
    W = \frac{\varepsilon_0}{2}\iiint_{\mathbb{R}^3}E^2\dmr{\tau}.
\end{equation}

\par\noindent\rule{\textwidth}{0.4pt}
\subsection{Conductors}
There are \textbf{five} fundamental properties of a conductors,
\begin{enumerate}
    \item $E = 0$ inside a conductor.
    \item $\rho = 0$ inside a conductor.
    \item Any net charge resides on the surface.
    \item A conductor is an equipotential.
    \item $E$ is perpendicular to the surface, just outside a conductor.
\end{enumerate}

Now, let's consider the force per unit area $\vec{f}$ on any charges that rests on the surface of the conductor, with references to page 102 in the book,\footnote{`The book' is exclusively used in this note to `Introduction to Electromagnetism' by David J. Griffths} we know that
\begin{equation}
    \vec{f} = \sigma \vec{E}_{\text{average}} = \frac12 \sigma (\vec{E}_{\text{above}}-\vec{E}_{\text{below}}).
\end{equation}
In particular, that for conductors, where we know $\vec{E}_{\text{below}} = 0$ (equipotential inside the conductor), we obtain 
\begin{equation}
    \vec{f} = \frac{1}{2\varepsilon_0}\sigma^2 \hat{n}
\end{equation} 
\par\noindent\rule{\textwidth}{0.4pt}

\subsection{Capacitors}
Note that the one thing we need to know about capacitance is that it is nothing but a ``proportionality" between the stored charge and the potential difference;
\begin{equation*}
    Q = CV.
\end{equation*}


\section{Special Techniques}
\subsection{Uniqueness Theorems}
\textbf{First uniqueness theorem:} The solution to Laplace's equation in some volume $\mathcal{V}$ is uniquely determined if $V$ is specified on the boundary surface $S$.

\textbf{Second uniqueness theorem:} In a volume $\mathcal{V}$ surrounded by conductors and containing a specified charge density $\rho$, the electric field is uniquely determined if the \textit{total charge} on each conductor is given. (The region as a whole can be bounded by another conductor, or else unbounded.)\cite{Griffiths:611579}
\par\noindent\rule{\textwidth}{0.4pt}
\subsection{Separation of Variables}
\subsubsection{Cylindrical Coordinate}
You can find the full derivation \href{http://faculty.washington.edu/blayneh/cylcoord.pdf}{\textbf{here.}} For those who just care about the result (of potential $V$) without $\hat{z}$ dependence
\begin{multline}
    V(s,\phi) = a_0 + b_0 \ln s\\
    + \sum_{k=1}^{\infty} \left[ s^k ( a_k \cos k \phi + b_k \sin k \phi) + s^{-k} (c_k \cos k \phi + d_k \sin k \phi) \right].
\end{multline}

\subsubsection{Spherical Coordinate}
Once again, you can find the full derivation \href{http://www.physics.usu.edu/Wheeler/EM3600/Notes11SeparationOfVariablesSpherical.pdf}{\textbf{here.}} And the general result for the potential $V(s, \theta, \phi)$ is
\begin{equation}
    V(s, \theta, \phi) = \sum_{l = 0}^{\infty} \sum_{l}^{m=-l} \left(A_{lm}r^l + \frac{B_{lm}}{r^{l+1}} \right) P_{l}^{m}(\cos \theta) \exp(im\phi)
\end{equation}
\par\noindent\rule{\textwidth}{0.4pt}
\subsection{Multipole Expansion}
\subsubsection{The Multipole Expansion}
Note that $P_n$ represents the $n^{\text{th}}$ Legendre Polynomial,
\begin{equation}
    V(\vec{r})=\frac{1}{4\pi\varepsilon_0} \sum_{n=0}^{\infty}\frac{1}{r^{n+1}}\int(r')^{n}P_n(\cos \theta')\rho(\vec{r}')\dmr{\tau'}.
\end{equation}

\begin{figure}[H]
    \centering
    \includegraphics[width =0.5\textwidth]{multipole.png}
\end{figure}

\subsubsection{The Dipole Term}
Since $P_0 = 1$, the dipole moment is then 
\begin{equation}
    \vec{p} = \int \vec{r'} \rho(\vec{r'}) \dmr{\tau'}.
\end{equation}

And the dipole (DIPOLE ONLY) contribution to the potential is
\begin{equation}
    V_{\text{dip}}(\vec{r}) = \frac{1}{4 \pi \varepsilon_0} \frac{\vec{p}\cdot\hat{r}}{r^2}.
\end{equation}

\subsubsection{The Electric Filed of a Dipole}
The electric field is simply 
\begin{equation}
    \vec{E}_{\text{dip}}(r,\theta)=\frac{p}{4 \pi \varepsilon_0 r^3}(2\cos \theta \; \hat{r} + \sin \theta \; \hat{\boldsymbol\theta}).
\end{equation}




\par\noindent\rule{\textwidth}{0.4pt}
\par\noindent\rule{\textwidth}{0.4pt}
\section{Electric Field in Matter}

\subsection{Polarisation}
The torque a dipole experiences due to a field $\vec{E}$ is 
\begin{equation}
    \vec{N} = \vec{p} \cross \vec{E}.
\end{equation}
And the force the dipole experiences is 
\begin{equation}
    \vec{F} = (\vec{p}\cdot\vec{\nabla})\vec{E}
\end{equation}
\par\noindent\rule{\textwidth}{0.4pt}

\subsection{Induced Dipoles}
Suppose we have an uniform electric field $\vec{E}$, and an atom (dipole) is in this field. This atom will now posses some dipole moment. The dipole moment is \textbf{most of the times} approximately proportional to the field.
\begin{equation}
    \vec{p} = \alpha \vec{E}.
\end{equation}
\par\noindent\rule{\textwidth}{0.4pt}
\subsection{Bound Charges}
Let's suppose we have a blob of polarised material, with dipole moment $\vec{p} = \vec{P}(\vec{r'})\dmr{\tau'}$ in each volume element. The total potential is 
\[
    V(\vec{r})= \frac{1}{4 \pi \varepsilon_0}\iiint_{\mathcal{V}}\frac{\hat{\curly{r}}\cdot \vec{P}(\vec{r'})}{\curly{r}^2}\dmr{\tau'}.
\]
And after some maths, we then obtain
\begin{equation}
    V = \frac{1}{4 \pi \varepsilon_0} \oiint_{\mathcal{S}} \frac{1}{\curly{r}} \vec{P}\cdot \dmr{\vec{a'}} - \frac{1}{4 \pi \varepsilon_0} \iiint_{\mathcal{V}} \frac{1}{\curly{r}} (\grad \cdot \vec{P}) \dmr{\tau'}.    
\end{equation}

Notice that 
\begin{align*}
    \underbrace{\sigma_{b} = \vec{P} \cdot \hat{n}}_{\text{Surface bound charge}}, &&\text{and}&& \underbrace{\rho_b = -\grad \cdot \vec{P}}_{\text{Volume bound charge}}.
\end{align*}

Those are called bound charges (surface bound charges \& volume bound charge), we can just find them to get the voltage of the polarised dialectic blob instead of calculating the massive integral.
\par\noindent\rule{\textwidth}{0.4pt}
\subsection{The Electric Displacement}
\subsubsection{Gauss's Law in the Presence of DIelectrics}
Blah blah blah there is something called the ``free charge" because we don't live in a perfect world. So the total charge density is now
\[
    \rho = \rho_b + \rho_f.
\]

And Gauss' law reads

\begin{align}
     && &\varepsilon_0 \grad \cdot \vec{E} = \rho = \rho_b + \rho_f = -\grad\cdot\vec{P} + \rho_f;\nonumber \\
    &\implies& &\rho_f = \grad\cdot\left(\varepsilon_0 \vec{E} + \vec{P}\right);\nonumber \\
    &\implies& &\vec{D} = \varepsilon_0 \vec{E} + \vec{P};\\
    &\implies& &\grad \cdot \vec{D} = \rho_f;\\
    &\implies& &\oiint_{\mathcal{S}} \vec{D} \cdot \dmr{\vec{a}} = Q_{f_{\text{enc}}}.
\end{align}

\textit{Trick}: In most of cases, $\vec{D}$ is determined exclusively by the free charge if we are dealing with symmetrical systems.
\subsubsection{Boundary Conditions}
\begin{align}
    D^{\bot}_{\text{above}}-D^{\bot}_{\text{below}} &= \sigma_f;\\
    \vec{D}^{\parallel}_{\text{above}}-\vec{D}^{\parallel}_{\text{below}} &= \vec{P}^{\parallel}_{\text{above}}-\vec{P}^{\parallel}_{\text{above}};\\
    E^{\bot}_{\text{above}}-E^{\bot}_{\text{below}} &= \frac{\sigma}{\varepsilon_0};\\
    \vec{E}^{\parallel}_{\text{above}}-\vec{E}^{\parallel}_{\text{below}} &=0.
\end{align}
\par\noindent\rule{\textwidth}{0.4pt}
\subsection{Linear Dielectrics}
Iff a space that is entirely filled with a homogeneous linear dielectric, then 
\begin{align*}
    \div \vec{D} = \rho_f &&\text{and}&& \curl \vec{D} =0.
\end{align*}

And, also
\[
    \vec{D} = \varepsilon_0 \vec{E}_{\text{vac}}
\]
where $\vec{E}_{\text{vac}}$ is the filed the same free charge would produce in the absence of dielectric.

So, we can conclude that
\begin{equation}
    \vec{E} = \frac{1}{\varepsilon} \vec{D} = \frac{1}{\varepsilon_r}\vec{E}_{\text{vac}}.
\end{equation}

Now, just for the sake of a very simple example, if we have a free charge $q$ that is embedded in a large dielectric, the field it produces is 
\begin{equation}
    \vec{E} = \frac{1}{4 \pi \varepsilon} \frac{q}{r^2} \hat{r}.
\end{equation}

\par\noindent\rule{\textwidth}{0.4pt}
\par\noindent\rule{\textwidth}{0.4pt}
\section{Magnetostatics}
\subsection{Currents}
The magnetic force on a segment of current-carrying wire is evidently 
\begin{equation}
    \vec{F}_{\text{mag}} = \int (\vec{v}\cross\vec{B}) \dmr{q} = \int (\vec{v}\cross\vec{B})\lambda\dmr{l} = \int(\vec{I}\cross\vec{B})\dmr{l}.
\end{equation}
And because $\vec{I}$ and $\dmr{\vec{l}}$ both point in the same direction, we then just have
\begin{equation}
    \vec{F}_{\text{mag}} =\int I(\dmr{\vec{l}}\cross \vec{B}).
\end{equation}

Here is this another thing that is quite useful which is called the surface current density $\vec{K}$, it is defined as the current per unit width-perpendicular-to-flow. Say we have surface charge density $\sigma$ and it has velocity $\vec{v}$, then
\begin{equation}
    \vec{K} = \sigma \vec{v}.
\end{equation}
And the force it will experience in a magnetic field will be
\begin{equation}
    \vec{F}_{\text{mag}}=\int (\vec{v} \cross \vec{B})\sigma \dmr{a} = \int (\vec{K}\cross \vec{B})\dmr{a}.
\end{equation}

Moving on, let's introduce another $\vec{J}$ called the volume current density, it is defined as the current per unit area-perpendicular-to-flow. And for some volume charge density $\rho$ and velocity $\vec{v}$, we have
\begin{equation}
    \vec{J} = \rho \vec{v},
\end{equation}
and the magnetic force on a volume current is therefore
\begin{equation}
    \vec{F}_{\text{mag}} = \int (\vec{v}\cross\vec{B})\rho \dmr{\tau} = \int (\vec{J}\cross \vec{B})\dmr{\tau}.
\end{equation}
\par\noindent\rule{\textwidth}{0.4pt}
\subsection{The Biot-Savart Law}
\subsubsection{The Magnetic Field of a Steady Current}
The magnetic filed of a steady line current is given by the \textbf{Biot-Savart law}:
\begin{equation}
    \vec{B}(\vec{r}) = \frac{\mu_0}{4\pi}\int\frac{\vec{I}\cross\hat{\curly{r}}}{\curly{r}^2}\dmr{l'} = \frac{\mu_0}{4\pi}I\int \frac{\dmr{\vec{l'}}\cross\hat{\curly{r}}}{\curly{r}^2}.
\end{equation}
\par\noindent\rule{\textwidth}{0.4pt}

\subsection{The Divergence and Curl of B}
As the title stated, the divergence and the curl of $\vec{B}$ are
\begin{align*}
    \curl\vec{B} = \mu_0 \vec{J}, && \divergence \vec{B} =0.
\end{align*}


\par\noindent\rule{\textwidth}{0.4pt}
\subsection{Applications of Ampepe's Law}
The equation for the curl of $\vec{B}$
\begin{equation}
    \curl B = \mu_0 \vec{J},
\end{equation}
is called Ampere's law, and it can also be written in the following form
\begin{equation}
    \int(\curl\vec{B})\cot\dmr{\vec{a}} = \oint \vec{B} \cdot \dmr{\vec{l}} = \mu_0 \int \vec{J} \cdot \dmr{\vec{a}} = \mu_0 I_{\text{enc}}.
\end{equation}
\par\noindent\rule{\textwidth}{0.4pt}
\subsection{Magnetic Vector Potential}
Just like the electrical potential, we can have a vector potential for the magnetic field and it is 
\begin{equation}
    \vec{B} = \curl \vec{A},
\end{equation}
where $\vec{A}$ is the vector potential in magnetostatics.

And since $\vec{A}$ is divergenceless, the Amepere's law for $\vec{A}$ is then 
\begin{equation}
    \laplacian\vec{A} = - \mu_0 \vec{J}.
\end{equation}

Now let's look at the vector potential for line, volume and surface current respectively
\begin{align}
    &\text{Line Current:} &&\vec{A} = \frac{\mu_0}{4 \pi} \int \frac{\vec{I}}{\curly{r}}\dmr{l'} = \frac{\mu_0 I }{4 \pi} \int \frac{1}{\curly{r}}\dmr{l'};\\
    &\text{Surface Current:} &&\vec{A} = \frac{\mu_0 }{4 \pi} \int \frac{\vec{K}}{\curly{r}}\dmr{a'};\\
    &\text{Volume Current} &&\vec{A} = \frac{\mu_0}{4 \pi}\int \frac{\vec{J}(\vec{r'})}{\curly{r}}\dmr{\tau'}.
\end{align}

\par\noindent\rule{\textwidth}{0.4pt}
\subsection{Summary And Magnetostatics Boundary Conditions}
\subsubsection{Boundary Condition}
The magnetic filed on a surface only have one component that is discontinuous, namely the component that is parallel to the surface and normal to the surface current $\vec{K}$. We can express this discontinuity simply
\begin{equation}
    \vec{B}_{\text{above}} - \vec{B}_{\text{below}} = \mu_0(\vec{K}\cross \hat{n})
\end{equation}
Similar to potential in electro static, the vector potential is continuous across any boundary:
\begin{equation}
    \vec{A}_{\text{above}} = \vec{A}_{\text{below}}.
\end{equation}

Further on, know that the flux is just the contour integral of $\vec{A}$
\begin{equation}
    \oint \vec{A} \cdot \dmr{l} = \int \vec{B} \cdot \dmr{\vec{a}} = \mathrm{\Phi}. 
\end{equation}
It is worth to note that the derivative of $\vec{A}$ inherits the discontinuity of $\vec{B}$:
\begin{equation}
    \pdv{\vec{A}_{\text{above}}}{n} -\pdv{\vec{A}_{\text{below}}}{n} = -\mu_0 \vec{K}.
\end{equation}


\par\noindent\rule{\textwidth}{0.4pt}
\subsection{Magnetic Dipoles}
The vector potential due to a magnetic dipole is 
\begin{equation}
    \vec{A}_{\text{dip}}(\vec{r}) = \frac{\mu_0 I}{4 \pi r^2} \oint (\hat{r}\cdot\vec{r'})\dmr{\vec{l'}}=\frac{\mu_0}{4\pi}\frac{\vec{m}\cross \hat{r}}{r^2},
\end{equation}
where $\vec{m}$ is the magnetic dipole moment:
\begin{equation}
    \vec{m} = I \vec{a}. 
\end{equation}











\par\noindent\rule{\textwidth}{0.4pt}
%%%%%%%%%%%%%%%%%%%%%%%%%%%%%%%%%%%%%%%%%%%%%%%%%%%%%%%%%%%%%%%%%
\newpage
\section*{References}
\addcontentsline{toc}{section}{\protect\numberline{}References}%
\printbibliography
[heading = none]

%%%%%%%%%%%%%%%%%%%%%%%%%%%%%%%%%%%%%%%%%%%%%%%%%%%%%%%%%%%%%%%%%%
\end{document}