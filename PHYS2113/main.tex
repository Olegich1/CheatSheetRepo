\documentclass[12pt,english]{article}
\usepackage[a4paper,bindingoffset=0.2in,%
            left=1in,right=1in,top=1in,bottom=1in,%
            footskip=.25in]{geometry}
\usepackage{amsfonts, amsmath, amssymb, physics}
\usepackage[utf8]{inputenc}
\usepackage[english]{babel}
\usepackage{indentfirst}
\usepackage[backend=biber, citestyle=nature]{biblatex}
\usepackage[nottoc]{tocbibind}
\usepackage{hyperref}
\hypersetup{
  colorlinks   = true, %Colours links instead of ugly boxes
  urlcolor     = blue, %Colour for external hyperlinks
  linkcolor    = blue, %Colour of internal links
  citecolor   = red %Colour of citations
}

%%%%%%%%%%%%%%%%%%%%% PREAMBLE %%%%%%%%%%%%%%%%%%%%%%%%
\setlength{\parskip}{1em}
\setlength{\parindent}{2em}
\setlength{\parskip}{0em}

\newcommand{\dmr}[1]{\, \mathrm{d}#1} %command for the straight d (hehe)
\newcommand{\intt}[2]{\int_{#1}^{#2}} %easy integral command


\newtheorem{definition}{Definition}[section]
\newtheorem{theorem}{Theorem}[section]
\newtheorem{remark}{Remark}[subsubsection]

\numberwithin{equation}{subsection}

\addbibresource{ref.bib}
%%%%%%%%%%%%%%%%%%%%%%%%%%%%%%%%%%%%%%%%%%%%%%%%%%%%%%%

\title{PHYS2113 Classical Mechanics}
\author{Joey Liang}
\date{May 2021}
\begin{document}
\maketitle

\newpage
\tableofcontents

\newpage
\section{Introduction to Lagrangian Mechanics}

\subsection{Action}
\begin{definition}[Action]
    Action, termed $A$, is defined as
    \begin{equation}
        A = \int_{t_0}^{t_1} L \dmr{t}.
    \end{equation}
    Where $L(q,\dot{q}) = T - V = \frac12 m \dot{q}^2 - V(q)$ is what we call the Lagrangian.
\end{definition}
\par Note that action represents the integral over time of the Lagrangian which can be thought as the motion of the object at some point of time.\cite{noauthor_action_2021}

\par\noindent\rule{\textwidth}{0.4pt}
\subsection{The Euler-Lagrange Equation}
\begin{definition}[The Euler-Lagrange Equation]
    The Euler-Lagrange equation for a system with a single degree of freedom is
    \begin{equation}
        \dv{t}\pdv{L}{\dot{q}} - \pdv{L}{q} = 0.
    \end{equation}
\end{definition}
\subsubsection{Derivation}
We want to find a generalised solution for the path that minimises the variational problem integral
\begin{equation}
    S = \int_{x_1}^{x_2} f[y(x),y'(x),x]\dmr{x}.
\end{equation}

First, let's start off by defining that the `right' path (path with the least action/that minimise the variational problem) to be $y = y(x)$, and the wrong path is just a variation of the right path known as $Y(y(x),\alpha,\eta(x)) = y(x)+\alpha\eta(x)$.

Since the end points of the right path and the wrong path are the same we get
\begin{equation}
    \eta(x_1) = \eta(x_2) = 0.\label{2}
\end{equation}

Now, the variational problem interval in terms of the wrong path $S_0$ would be
\begin{align}
    S_0 & = \intt{x_1}{x_2} f(Y,Y',x) \dmr{x} \nonumber               \\
        & = \intt{x_1}{x_2} f(y+\alpha \eta, y' + \alpha \eta', x)dx.
\end{align}

Note that the only difference between integral $S$ and $S_0$ is the dependence on $\alpha$. So, ideally to minimise this integral, we would find the stationary point of $S_0$ in terms of $\alpha$, this can be expressed as
\begin{align}
    \dv{S_0}{\alpha} & = 0 \nonumber                                                                         \\
                     & = \intt{x_1}{x_2} \pdv{f}{\alpha} \dmr{x} \nonumber                                   \\
                     & = \intt{x_1}{x_2} \left(\eta \pdv{f}{y} + \eta' \pdv{f}{y'} \right) \dmr{x}.\label{1}
\end{align}

And now, using \eqref{2} and integration by parts, we obtain the following
\begin{equation}
    \intt{x_1}{x_2}\eta(x)\left(\pdv{f}{y}-\dv{x}\pdv{f}{y'}\right) = 0.
\end{equation}
For non-trivial solution, we must have
\[
    \pdv{f}{y}-\dv{x}\pdv{f}{y'} = 0,
\]
which is the Euler-Lagrange Equation.

\begin{remark}
    Note that for \eqref{1}, the following needs to be true for $Y = y + \alpha \eta$;
    \[
        \pdv{}{Y} f(Y,Y',x) = \pdv{}{y} f(y+\alpha \eta, y'+\alpha \eta', x).
    \]
    The proof is simple,
    \begin{align*}
        \pdv{}{y} f(y+\alpha \eta, y'+\alpha \eta', x) & = f'(y+\alpha \eta, y'+\alpha \eta', x) \\
                                                       & = f'(Y, Y', x).
    \end{align*}
\end{remark}

\subsubsection{Example}
Refer to Taylor's\footnote{The book `Classical Mechanics'\cite{TaylorJohnR.JohnRobert2005Cm} By John R. Taylor} \textbf{Example 6.2} on page 222.

\par\noindent\rule{\textwidth}{0.4pt}
\subsection{Proof of Lagrange's Equations with Constraints}
Refer to \textbf{Section 7.4} (pg 250) in Classical Mechanics by JR Taylor.

One important thing: the right path of any action must follows Newton's second law.

\par\noindent\rule{\textwidth}{0.4pt}
\subsection{Lagrangian Multipliers}
Suppose we have a system with two variables $x, y$. Which are linked together with a constraint, say $F(x,y) = C \in \mathbb{R}.$

Now, let us introduce this function $\lambda(t)$ (aka Lagrange multiplier), and now the E-L Equation for $x$ becomes
\begin{equation}
    \pdv{L}{x} +\lambda(t)\pdv{F}{x} = \dv{t}\pdv{L}{\dot{x}},\label{3}
\end{equation}
similarly, the E-L equation for $y$ is
\begin{equation}
    \pdv{L}{y} +\lambda(t)\pdv{F}{y} = \dv{t}\pdv{L}{\dot{y}}.\label{4}
\end{equation}

Now, fingers cross that we can solve the above two equations. Then we can rewrite one of the variable in terms of the other and obtain their time derivatives, e.g.
\begin{align}
    x                 & = g(y),          \\
    \implies \dot{x}  & = \dv{t}g(y),    \\
    \implies \ddot{x} & = \dv[2]{t}g(y).
\end{align}

And then we can substitute them into our results from Eq. \eqref{3} \& \eqref{4}. If we don't drink and do maths, by now, we should just be able to obtain $\lambda(t)$ with a little more algebra.

\par\noindent\rule{\textwidth}{0.4pt}
\subsection{Conservation of Energy and the Hamiltonian}
Refer to section \textbf{Section 7.8} (pg 269) in Taylor's for derivation.

\par\noindent\rule{\textwidth}{0.4pt}
\subsection{Legendre Transform}
Let's have and input system that consists of
\begin{align*}
     & F(u_1, \dots u_n, w_1, \dots w_n); \\
     & v_i = \pdv{F}{u_i};                \\
     & G = \sum_{i = 1}^{n} u_i v_i - F.
\end{align*}

By performing a Legendre Transform, we will get the output system that consists of
\begin{align*}
     & G(v_1, \dots v_n, w_1, \dots w_n); \\
     & u_i = \pdv{G}{v_i};                \\
     & F = \sum_{i = 1}^{n} u_i v_i - G.
\end{align*}

\par\noindent\rule{\textwidth}{0.4pt}
\par\noindent\rule{\textwidth}{0.4pt}
\section{Very Short Session on Hamiltonian Mechanics}
\subsection{From Lagrangian to Hamiltonian}
Let us record that in the Lagrangian, we have a function of three dependents,
\begin{equation}
    L = L(q_1, \dots q_n, \dot{q}_1, \dots \dot{q}_n, t) = T - U,
\end{equation}
and obviously, the Euler-Lagrange equation would be
\begin{equation}
    \pdv{L}{q_i} = \dv{t}\pdv{L}{\dot{q}_i}.
\end{equation}
Also remember that the \textbf{generalised momentum} (a.k.a canonical momentum/momentum conjugate to $q_i$) is given by
\begin{equation}
    p_i = \pdv{L}{\dot{q}_i}.
\end{equation}

The Hamiltonian is defined as
\begin{equation}
    H = \sum_{i = 1}^{n} p_i \dot{q}_i - L.
\end{equation}

If we apply the Legendre Transform to the E-L equation, we will find the Hamilton's equations
\begin{align}
    \dot{q}_i = \pdv{H}{p_i}, &  & \dot{p}_i = -\pdv{H}{q_i}.
\end{align}

\par\noindent\rule{\textwidth}{0.4pt}
\subsection{Poisson Brackets}
Definition of Poisson Brackets in \textbf{canonical coordinates} $(q_i, p_i)$ on the phase space, given two functions $f(p_i, q_i, t)$ and $g(p_i, q_i, t),$ the Poisson bracket take the form \cite{noauthor_poissonbracket_2021}
\begin{equation}
    \{f,g\} = \sum_{i = 1}^{N} \left(\pdv{f}{q_i}\pdv{g}{p_i}-\pdv{f}{p_i}\pdv{g}{q_i}\right).
\end{equation}
The Poisson brackets of the canonical coordinates are
\begin{align*}
     & \{q_i,q_j\}=0            \\
     & \{p_i,p_j\}=0            \\
     & \{q_i,p_j\}=\delta_{ij}.
\end{align*}

\par\noindent\rule{\textwidth}{0.4pt}
\par\noindent\rule{\textwidth}{0.4pt}
\section{Harmonic Oscillator}
\subsection{Damped Harmonic Oscillators}
Suppose we have a system that can be described by the following ODE
\begin{equation}
    m\ddot{x} + b \dot{x} + kx = 0.
\end{equation}
Where $m$ denotes the mass, $b$ denotes resistance and $k$ denotes the spring constant. Now let's define the damping constant to be
\begin{equation*}
    \beta = \frac{b}{2m}.
\end{equation*}

And, the system will have a \textbf{natural frequency} (the frequency at which it would oscillate if there were no resistive force present)
\begin{equation*}
    \omega_0 = \sqrt{\frac{k}{m}}.
\end{equation*}

The general solution of this system is
\begin{equation}
    x(t) = e^{-\beta t}\left(C_1 e^{\sqrt{\beta^2-\omega_0^2}t} + C_2 e^{-\sqrt{\beta^2-\omega_0^2}t}\right)
\end{equation}

\par\noindent\rule{\textwidth}{0.4pt}
\subsection{Undamped Oscillators}
I hope it is clear that for an undamped oscillator, we just need to take the damping constant $\beta$ to be zero, so the EOM for the system would be
\begin{equation}
    x(t) = C_1 e^{\omega_0 t} + C_2 e^{-\omega_0 t}.
\end{equation}

\par\noindent\rule{\textwidth}{0.4pt}
\subsection{Weak Damping}
In the case of weakly damped oscillators, damping constant $\beta$ is said to be
\begin{equation*}
    \beta < \omega_0.
\end{equation*}

Now the EOM becomes
\begin{equation}
    x(t) = e^{-\beta t}\left(C_1 e^{i \omega_1 t} + C_2 e^{-i \omega_1 t}\right),
\end{equation}
where
\begin{equation*}
    \omega_1 = \sqrt{\omega_0^2 - \beta^2}.
\end{equation*}

Note, in the case of underdamping. $\beta$ is said to be the decay parameter, since $1/\beta$ the time taken for the oscillator to fall to $1/e$ of its original amplitude.
\begin{equation*}
    \text{(decay parameter)} = \beta.
\end{equation*}

\par\noindent\rule{\textwidth}{0.4pt}
\subsection{Critical Damping}
Critical damping is said to be happening when
\begin{equation*}
    \beta = \omega_0.
\end{equation*}
The general solution for critical damping is
\begin{equation}
    x(t) = C_1 e^{-\beta t} + C_2 t e^{-\beta t},
\end{equation}
with decay parameter
\begin{equation*}
    \text{(decay parameter)} = \beta = \omega_0.
\end{equation*}

\par\noindent\rule{\textwidth}{0.4pt}
\subsection{Strong Damping}
Over damping is said to be happening when
\begin{equation*}
    \beta > \omega_0.
\end{equation*}
The general EOM is
\begin{equation}
    x(t) = C_1 e^{-\left(\beta - \sqrt{\beta^2 - \omega_0^2}\right)t} + C_2 e^{-\left(\beta + \sqrt{\beta^2 - \omega_0^2}\right)t},
\end{equation}
with decay parameter
\begin{equation*}
    \text{(decay parameter)} = \beta - \sqrt{\beta^2 - \omega_0^2}.
\end{equation*}

\par\noindent\rule{\textwidth}{0.4pt}
\subsection{Forced Damped Oscillators}






















%%%%%%%%%%%%%%%%%%%%%%%%%%%%%%%%%%%%%%%%%%%%%%%%%%%%%%%%%%%%%%%%%%%%%%%%%%%%%%%%%%%%%%%%%%%%%%%%%%%%%%%%%%%%%%%%%%%%%%%%%%%%%%%%%%%%%%%%%%%%%%%%%%%%%%%%%%%%%%%%%%%%%%%%%%%%%%%%%%

\newpage
\section*{References}
\addcontentsline{toc}{section}{\protect\numberline{}References}
\printbibliography[heading = none]
%%%%%%%%%%%%%%%%%%%%%%%%%%%%%%%%%%%%%%%%%%%%%%%%%%%%%%%%%%%%%%%%%%%%%%%%%%%%%%%%%%%%%%%%%%%%%%%%%%%%%%%%%%%%%%%%%%%%%%%%%%%%%%%%%%%%%%%%%%%%%%%%%%%%%%%%%%%%%%%%%%%%%%%%%%%%%%%%%%
\end{document}
