\documentclass[a4paper]{article}
\usepackage{amsfonts, amsmath, amssymb, physics}
\usepackage[utf8]{inputenc}
\usepackage[english]{babel}
\usepackage{indentfirst}
\usepackage[backend=biber, citestyle=nature]{biblatex}
\usepackage[nottoc]{tocbibind}
\addbibresource{ref.bib}

%%%%%%%%%%%%%%%%%%%%% PREAMBLE %%%%%%%%%%%%%%%%%%%%%%%%
\setlength{\parindent}{2em}
\setlength{\parskip}{0em}
\newcommand{\dmr}[1]{\, \mathrm{d}#1} %command for the straight d (hehe)
\newtheorem{definition}{Definition}[section]
\newtheorem{theorem}{Theorem}[section]

%%%%%%%%%%%%%%%%%%%%%%%%%%%%%%%%%%%%%%%%%%%%%%%%%%%%%%%

\title{PHYS2113 Classical Mechanics}
\author{Joey Liang}
\date{May 2021}
\begin{document}
\maketitle

\newpage
\tableofcontents

\newpage
\section{Introduction to Lagrangian Mechanics}

\subsection{Action}
\begin{definition}[Action]
    Action, termed $A$, is defined as
    \begin{equation}
        A = \int_{t_0}^{t_1} L \dmr{t}.
    \end{equation}
    Where $L(q,\dot{q}) = T - V = \frac12 m \dot{q}^2 - V(q)$ is what we call the Lagrangian.
\end{definition}
\par Note that action represents the integral over time of the Lagrangian which can be thought as the motion of the object at some point of time.\cite{noauthor_action_2021}

\subsection{The Euler-Lagrange Equation}
\begin{definition}[The Euler-Lagrange Equation]
    The Euler-Lagrange equation for a system with a single degree of freedom is
    \[
        \dv{t}\pdv{L}{\dot{q}} - \pdv{L}{q} = 0.
    \]
\end{definition}

\subsubsection{Example}
Refer to Taylor's\footnote{The book `Classical Mechanics' By John R. Taylor} \textbf{Example 6.2} on page 222.















%%%%%%%%%%%%%%%%%%%%%%%%%%%%%%%%%%%%%%%%%%%%%%%%%%%%%%%%%%%%%%%%%%%%%%%%%%%%%%%%%%%%%%%%%%%%%%%%%%%%%%%%%%%%%%%%%%%%%%%%%%%%%%%%%%%%%%%%%%%%%%%%%%%%%%%%%%%%%%%%%%%%%%%%%%%%%%%%%%
\newpage
\section*{References}
\addcontentsline{toc}{section}{\protect\numberline{}References}%
\printbibliography[heading = none]
%%%%%%%%%%%%%%%%%%%%%%%%%%%%%%%%%%%%%%%%%%%%%%%%%%%%%%%%%%%%%%%%%%%%%%%%%%%%%%%%%%%%%%%%%%%%%%%%%%%%%%%%%%%%%%%%%%%%%%%%%%%%%%%%%%%%%%%%%%%%%%%%%%%%%%%%%%%%%%%%%%%%%%%%%%%%%%%%%%
\end{document}
