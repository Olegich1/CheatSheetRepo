\documentclass[a4paper]{article}
\usepackage{amsfonts, amsmath, amssymb, physics}
\usepackage[utf8]{inputenc}
\usepackage[english]{babel}
\usepackage{indentfirst}
\usepackage[backend=biber, citestyle=nature]{biblatex}
\usepackage[nottoc]{tocbibind}
\usepackage{halloweenmath}


%%%%%%%%%%%%%%%%%%%%% PREAMBLE %%%%%%%%%%%%%%%%%%%%%%%%
\setlength{\parindent}{2em}
\setlength{\parskip}{0em}

\newcommand{\dmr}[1]{\, \mathrm{d}#1} %command for the straight d (hehe)
\newcommand{\intt}[2]{\int_{#1}^{#2}} %easy intergral command


\newtheorem{definition}{Definition}[section]
\newtheorem{theorem}{Theorem}[section]
\newtheorem{remark}{Remark}[subsubsection]

\numberwithin{equation}{subsection}

\addbibresource{ref.bib}
%%%%%%%%%%%%%%%%%%%%%%%%%%%%%%%%%%%%%%%%%%%%%%%%%%%%%%%

\title{PHYS2113 Classical Mechanics}
\author{Joey Liang}
\date{May 2021}
\begin{document}
\maketitle

\newpage
\tableofcontents

\newpage
\section{Introduction to Lagrangian Mechanics}

\subsection{Action}
\begin{definition}[Action]
    Action, termed $A$, is defined as
    \begin{equation}
        A = \int_{t_0}^{t_1} L \dmr{t}.
    \end{equation}
    Where $L(q,\dot{q}) = T - V = \frac12 m \dot{q}^2 - V(q)$ is what we call the Lagrangian.
\end{definition}
\par Note that action represents the integral over time of the Lagrangian which can be thought as the motion of the object at some point of time.\cite{noauthor_action_2021}

\subsection{The Euler-Lagrange Equation}
\begin{definition}[The Euler-Lagrange Equation]
    The Euler-Lagrange equation for a system with a single degree of freedom is
    \begin{equation}
        \dv{t}\pdv{L}{\dot{q}} - \pdv{L}{q} = 0.
    \end{equation}
\end{definition}

\subsubsection{Derivation}
We want to find a generalised solution for the path that minimises the variational problem integral
\begin{equation}
    S = \int_{x_1}^{x_2} f[y(x),y'(x),x]\dmr{x}.
\end{equation}

First, let's start off by defining that the `right' path (path with the least action/that minimise the variational problem) to be $y = y(x)$, and the wrong path is just a variation of the right path known as $Y(y(x),\alpha,\eta(x)) = y(x)+\alpha\eta(x)$. 

Since the end points of the right path and the wrong path are the same we get
\begin{equation}
    \eta(x_1) = \eta(x_2) = 0.\label{2}    
\end{equation}

Now, the vairational problem interval in terms of the wrong path $S_0$ would be
\begin{align}
    S_0 &= \intt{x_1}{x_2} f(Y,Y',x) \dmr{x} \nonumber \\
    &= \intt{x_1}{x_2} f(y+\alpha \eta, y' + \alpha \eta', x)dx.
\end{align}

Note that the only difference between integral $S$ and $S_0$ is the dependence on $\alpha$. So, idealy to minimise this integral, we would find the stationary point of $S_0$ in terms of $\alpha$, this can be expressed as 
\begin{align}
    \dv{S_0}{\alpha} &= 0 \nonumber \\
    &= \intt{x_1}{x_2} \pdv{f}{\alpha} \dmr{x} \nonumber \\
    &= \intt{x_1}{x_2} \left(\eta \pdv{f}{y} + \eta' \pdv{f}{y'} \right) \dmr{x}.\label{1}
\end{align}

And now, using \eqref{2} and integration by parts, we obtain the following
\begin{equation}
    \intt{x_1}{x_2}\eta(x)\left(\pdv{f}{y}-\dv{x}\pdv{f}{y'}\right) = 0.
\end{equation}
For non-trivial solution, we must have 
\[
    \pdv{f}{y}-\dv{x}\pdv{f}{y'} = 0,
\]
which is the Euler-Lagrange Equation.

\begin{remark}
    Note that for \eqref{1}, the following needs to be true for $Y = y + \alpha \eta$;
    \[
        \pdv{}{Y} f(Y,Y',x) = \pdv{}{y} f(y+\alpha \eta, y'+\alpha \eta', x).
    \]
    The proof is simple,
    \begin{align*}
        \pdv{}{y} f(y+\alpha \eta, y'+\alpha \eta', x) &= f'(y+\alpha \eta, y'+\alpha \eta', x) \\
        &= f'(Y, Y', x).
    \end{align*}
\end{remark}

\subsubsection{Example}
Refer to Taylor's\footnote{The book `Classical Mechanics'\cite{TaylorJohnR.JohnRobert2005Cm} By John R. Taylor} \textbf{Example 6.2} on page 222.
















%%%%%%%%%%%%%%%%%%%%%%%%%%%%%%%%%%%%%%%%%%%%%%%%%%%%%%%%%%%%%%%%%%%%%%%%%%%%%%%%%%%%%%%%%%%%%%%%%%%%%%%%%%%%%%%%%%%%%%%%%%%%%%%%%%%%%%%%%%%%%%%%%%%%%%%%%%%%%%%%%%%%%%%%%%%%%%%%%%
\newpage
\section*{References}
\addcontentsline{toc}{section}{\protect\numberline{}References}%
\printbibliography[heading = none]
%%%%%%%%%%%%%%%%%%%%%%%%%%%%%%%%%%%%%%%%%%%%%%%%%%%%%%%%%%%%%%%%%%%%%%%%%%%%%%%%%%%%%%%%%%%%%%%%%%%%%%%%%%%%%%%%%%%%%%%%%%%%%%%%%%%%%%%%%%%%%%%%%%%%%%%%%%%%%%%%%%%%%%%%%%%%%%%%%%
\end{document}
