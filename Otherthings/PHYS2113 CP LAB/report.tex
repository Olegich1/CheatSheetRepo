\documentclass[aps,prl,reprint,10pt,amsmath,amssymb,superscriptaddress,a4paper]{revtex4-2}
\usepackage{xurl} % handle line breaks in long URL strings gracefully
\usepackage[utf8]{inputenc} % dding UNICODE support
\usepackage{indentfirst} % Indentation always (personal preference)
\usepackage{hyperref} % Hyper link for everything
\usepackage{graphicx} % Graphics package
\usepackage{dcolumn} % Double column package
\usepackage{amsmath, amsfonts, amsthm, physics} % Maths packages
\usepackage[margin=2cm]{geometry} % Sets 2cm margins
\usepackage{datetime} % Package for automatic date & time
\usepackage{enumitem} % Enumeration Package
\usepackage{listings} % To list code in your lab report
\usepackage{appendix} % This one kinda explains itself
\usepackage{float} % So the figures will behave
\usepackage{natbib}

\begin{document}
\title{Coupled Pendula Lab Report}
\author{J. Liang (z5261830)}
\affiliation{Cohort B - Mon 9-12 class}
\affiliation{Word count: XXXX words}
\date{\currenttime~\today}
\begin{abstract}
    This report presents the observation and interpretations of the behaviour of coupled pendula.
\end{abstract}

\maketitle
\section{INTRODUCTION}
The behaviour of harmonic oscillators has been a long time interest for mathematicians and physicists. In this lab, I am going to present my observation and interpretations of the two normal modes and one of their linear combination mode (beat mode) of such a system with experimental data.

\section{AIM}
In this lab, the aim is to use the obtained data from the inphase oscillation, out of phase oscillation, and the beat mode, to quantitatively characterise the behaviours of coupled pendula system. Hence forward, establish a connection between this coupled system and its predecessor (a simple uncoupled pendulum) and its successor (many pendula all coupled together).

\section{METHOD}






































%\bibliographystyle{agsm}
%%%%%%%%%%%%%%%%%%%%%%%%%%%%%%%%%%%%%%%%%%%%%%%%%
\onecolumngrid
\newpage
\appendix
\section{Appendix}
\subsection{Figures}





\end{document}