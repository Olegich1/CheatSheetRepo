\documentclass[a4paper]{article}
\usepackage{amsfonts, amsmath, amssymb, physics}
\usepackage[utf8]{inputenc}
\usepackage[english]{babel}
\usepackage{indentfirst}

%%%%%%%%%%%%%%%%%%%%% PREAMBLE %%%%%%%%%%%%%%%%%%%%%%%%
\setlength{\parindent}{2em}
\setlength{\parskip}{0em}
\newcommand{\dmr}[1]{\, \mathrm{d}#1} %command for the straight d (hehe)
%%%%%%%%%%%%%%%%%%%%%%%%%%%%%%%%%%%%%%%%%%%%%%%%%%%%%%%

\title{PHYS2111 Cheat Sheet}
\author{Moisty Eyes}
\date{April 2021}
\begin{document}
\maketitle
\section{Formula Sheet}
\subsection{Expectation Value}


\par Expectation value of function \(f(x)\) subject to \(\Psi(x,t)\)
\[
    \mathbb{P}(f(x))=\int_{-\infty}^{\infty}f(x)\norm{\Psi(x,t)}^{2}\dmr{x}.
\]
Note that the Hamiltonian operator $H$ is 
\[
    H = -\frac{\hbar^2}{2m}\pdv[2]{x} + V(x,t),    
\]
and the expectation value of the energy $\langle H \rangle$ is
\begin{align*}
    \langle H \rangle &= \left\langle -\frac{\hbar^2}{2m}\pdv[2]{x} + V(x,t) \right\rangle\\
    &= \int_{-\infty}^{\infty} \Psi^{*}(x,t) \left(-\frac{\hbar^2}{2m}\pdv[2]{x} + V(x,t)\right) \Psi(x,t)\dmr{x}.
\end{align*}
\par General case, the expectation value for any operator $Q(x,\hat{p})$ is
\[
    \int_{-\infty}^{\infty} \Psi^{*}(x,t) Q(x,\hat{p}) \Psi(x,t) \dmr{x}.     
\]

\subsection{Poisition and Momentum Operators}
\par The position operator,
\[
    \hat{x} = x.
\]
\par The momentum operator,
\[
    \hat{p} = -i\hbar \pdv{x}.
\]

\subsection{The Infinite Square Well}
\par Time Dependent Schrodinger's Equation
\[
    \Psi(x,t) =  \sum_{n=1}^{\infty}c_{n}\sqrt{\frac{2}{a}}\sin\left(\frac{n\pi}{a}x\right)\exp\left(-i E_{n} t\right).
\]
with constant
\[
    c_n = \sqrt{\frac{2}{a}}\int_{0}^{a}\sin\left(\frac{n\pi}{a}x\right)\Psi(x,0)\dmr{x}.
\]
and allowed energy
\[
    E_{n} = \frac{n^2\pi^2\hbar^2}{2ma^2}.
\]

\subsection{The Harmonic Oscillator}
\par A harmonic oscillator has potential energy
\begin{equation*}
    V(x) = \frac12 m \omega^2 x^2,
\end{equation*}
With ground state $\psi_{0}(x)$ 
\[
    \psi_0(x)=\left(\frac{m \omega}{\pi \hbar}\right)^{\frac{1}{4}}\exp\left(-\frac{m \omega}{2\hbar}x^2\right).
\]
The ladder operator
\[
    a_{\pm}=\frac{1}{\sqrt{2\hbar m \omega}}(\mp i \hat{p}+m\omega x)\tag{$\omega = \sqrt{\frac{k}{m}}$},
\]
and to extract the n-th state with the ladder operator
\[
    \psi_{n}(x)=A_n(\hat{a}_{+})^{n}\psi_{0}(x),
\]
with 
\[
    E_{n} = \left(n+\frac12\hbar\omega\right).   
\]
\subsection{The Free Particle}
The initial condition can be expressed in the Fourier $k$ space,
\[
    \phi(k)=\frac{1}{\sqrt{2\pi}}\int_{-\infty}^{\infty}\Psi(x,0)\exp(-ikx)\dmr{x} \tag{\(k = \frac{\sqrt{2mE}}{\hbar}\)},
\]
we can then use this $\phi(k)$ calculated above to determine the time dependent wave equation for the free particle
\[
    \Psi(x,t)=\frac{1}{\sqrt{2\pi}}\int_{-\infty}^{\infty}\phi(k)\exp\left(i\left(kx-\frac{\hbar k^2}{2m}t\right)\right)\dmr{k}.
\]
Notably, the particle has a phase velocity and a group velocity,
\begin{align*}
    v_{\text{phase}}=\frac{\omega}{k}, && v_{\text{group}}=\dv{\omega}{k}.\tag{$\omega = \frac{\hbar k^2}{2m}$}
\end{align*}
Also note that 
\[
    v_{\text{group}} = v_{\text{classical}}.
\]

\par For a free particle, note that the potential energy is zero namely \(V(x,t)=0\). Also note that we can switch from the Fourier $k$ space and the momentum $p$ space (\(p = \hbar k\)), obtaining the following
\[
    \phi(p) = \frac{1}{\sqrt{2\pi\hbar}}\int_{-\infty}^{\infty}\Psi(x,0)\exp(-\frac{ip}{\hbar}x)\dmr{x},
\]
and
\[
    \Psi(x,t) = \frac{1}{\sqrt{2 \pi\hbar }}\int_{-\infty}^{\infty} \phi(p) \exp \left(\frac{ip}{\hbar}x\right)\exp\left(\frac{-i E t}{\hbar}\right) \dmr{p}\tag{$E = \frac{\hbar^2k^2}{2m}$}
\]  

\subsection{The Delta-Function Potential}
\par The delta function well has \textbf{only one} bound state (for $\alpha > 0$) namely
\[
    \phi(x) = \frac{\sqrt{m\alpha}}{\hbar}\exp\left(\frac{-m\alpha\abs*{x}}{\hbar}\right),    
\]
with \textbf{only one} allow energy
\[
    E = -\frac{m\alpha^2}{2\hbar^2}.
\]

\par The reflection coefficient $R$ and the transmission coefficient $T$ can be expressed as the following
\begin{align*}
    R = \frac{1}{1+\frac{2\hbar^2E}{m\alpha^2}}, &&  T = \frac{1}{1+\frac{m\alpha^2}{2\hbar^2E}}.
\end{align*}

\subsection{The Finite Squre Well}
\par The finite squre well has the potential function such that (note that this is a flipped tophat function),
\[
    V(x) = 
        \begin{cases}
            -V_{0} & \text{if } -a \leq x \leq a,\\
            0 & \text{if } \abs*{x} > a.
        \end{cases}    
\]
\par For scattering states, the energy for perfect transmission is given by 
\[
    E_{n}+V_{0} = \frac{n^2\pi^2\hbar^2}{2m(2a)^2}.
\]  
Note that is also the allowed eneries for the infinite square well.

\newpage
\section{Formalism}
\subsection{Hermitian Matrices}
\begin{flushleft}
    \textbf{Properties of Hermitian Matrices}:
\end{flushleft} 
\begin{enumerate}
    \item The diagonal elements are real, as they must be their complex conjugate.
    \item The off-diagonal symmetric pairs must be complex conjugates $m_{ij} = m_{ji}^{*}$. If they are real, then they will be equal.
    \item Hermitian matrices are \textbf{normal}, i.e. $M^{\dagger}M = MM^{\dagger}$, and therefore \textbf{diagonalizable}, which means they can be transformed such that all off-diagonal elements are zero.
    \item The sum of any two Hermitian matrices is also Hermitian.
    \item The determinant of a Hermitian matrix is real.
\end{enumerate}

\subsection{Fundamental Theorem of Quantum Mechanics}
\begin{flushleft}
    \textbf{The Fundamental Theorem}
\end{flushleft}
\begin{enumerate}
    \item If $\lambda_1$ and $\lambda_2$ are two unequal eigenvalues of a Hermitian operator, then the corresponding eigenvectors are orthogonal.
    \item Even if $\lambda_1 = \lambda_2$, the coreresponding eigenvectors can be chosen to be orthogonal. We use the term degeneracy to describe the case where two different eigenvectors have the same eigenvalue. $\lambda_1$ and $\lambda_2$ are referred to as degenerate.
    \item The eigenvectors of a Hermitian operator are a complete set, i.e. any vector the operator can generate can be expanded as a sum of its eigenvectors.
\end{enumerate}

\par A \textbf{distillation} of the above: For any observable, we have an operator, the eigenvectors of that operator will be the basis for the vector space we operate in.

\begin{flushleft}
    \textbf{The Principles}
\end{flushleft}
\begin{enumerate}
    \item The observable or measurable quantities of quantum mechanics are represented by linear opertors $\mathbf{L}$, with $\lambda_i,\, \ket{\lambda_i}$ as its eigenvalue and eigenvector respectively.
    \item The possible results of a measuremeant are the eigenvalues of the operator that represents the observable.
    \item unambiguously distinguishable states are representsed by orthogonal vectors.
    \item if $\ket{A}$ is the state-vector of a system, and the observable $\mathbf{L}$ is measured, the probability to observe value $\lambda_i$ is 
    \[
        \mathbb{P}(\lambda_i) = \norm{\bra{A}\ket{\lambda_i}}^2=\bra{A}\ket{\lambda_i}\bra{\lambda_i}\ket{A}    
    \]
\end{enumerate}

\subsection{Recalling on Statistics}
\subsubsection{Statitical Correlation}
Consider 
\[
    P = \langle \sigma_A \sigma_B \rangle - \langle \sigma_A \rangle \langle \sigma_B \rangle,
\]
if $P \neq 0,$ then $\langle \sigma_A \rangle$ unrelated to $\langle \sigma_B \rangle$, else they are related.\\

\par For event $a$ and $b$ to be independent,
\[
    \mathbb{P}(a,b) = \mathbb{P}(a) \mathbb{P}(b).
\]  
\subsubsection{The Cauchy-Swarz Inequality}
It states that 
\[
    \braket{A}{A}\braket{B}{B} \geq \abs*{\braket*{A}{B}}^2.
\]  
\subsection{Commutators}
For some $[L,M]=LM-ML$, if $[L, M] = 0$, then $L$ and $M$ commute and therefore there can exist a zero undertainty (we can know both of the observable precisely).

















%%%%%%%%%%%%%%%%%% TO DO LIST %%%%%%%%%%%%%%%%%%%%%%
%%%%%%%%%%%%%%%%%%%%%%%%%%%%%%%%%%%%%%%%%%%%%%%%%%%%
\end{document}